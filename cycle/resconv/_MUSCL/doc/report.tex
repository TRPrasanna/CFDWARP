\documentclass{warpdoc}
\newlength\lengthfigure                  % declare a figure width unit
\setlength\lengthfigure{0.158\textwidth} % make the figure width unit scale with the textwidth
\usepackage{psfrag}         % use it to substitute a string in a eps figure
\usepackage{subfigure}
\usepackage{rotating}
\usepackage{pstricks}
\usepackage[innercaption]{sidecap} % the cute space-saving side captions
\usepackage{scalefnt}
\usepackage{amsbsy}
\usepackage{bm}
\usepackage{amsmath}

%%%%%%%%%%%%%=--NEW COMMANDS BEGINS--=%%%%%%%%%%%%%%%%%%%%%%%%%%%%%%%%%%
\newcommand{\alb}{\vspace{0.1cm}\\} % array line break
\newcommand{\mfa}{\scriptscriptstyle}
\newcommand{\mfb}{\scriptstyle}
\newcommand{\mfc}{\textstyle}
\newcommand{\mfd}{\displaystyle}
\newcommand{\hlinex}{\vspace{-0.34cm}~~\\ \hline \vspace{-0.31cm}~~\\}
\newcommand{\hlinextop}{\vspace{-0.46cm}~~\\ \hline \hline \vspace{-0.32cm}~~\\}
\newcommand{\hlinexbot}{\vspace{-0.37cm}~~\\ \hline \hline \vspace{-0.50cm}~~\\}
\newcommand{\tablespacing}{\vspace{-0.4cm}}
\newcommand{\fontxfig}{\footnotesize\scalefont{0.918}}
\newcommand{\fontgnu}{\footnotesize\scalefont{0.896}}
\renewcommand{\fontsizetable}{\footnotesize\scalefont{1.0}}
\renewcommand{\fontsizefigure}{\footnotesize}
%\renewcommand{\vec}[1]{\pmb{#1}}
%\renewcommand{\vec}[1]{\boldsymbol{#1}}
\renewcommand{\vec}[1]{\bm{#1}}
\setcounter{tocdepth}{3}
\let\citen\cite
\newcommand\frameeqn[1]{\fbox{$\displaystyle #1$}}

%%%%%%%%%%%%%=--NEW COMMANDS BEGINS--=%%%%%%%%%%%%%%%%%%%%%%%%%%%%%%%%%%

\setcounter{tocdepth}{3}

%%%%%%%%%%%%%=--NEW COMMANDS ENDS--=%%%%%%%%%%%%%%%%%%%%%%%%%%%%%%%%%%%%



\author{
  Bernard Parent
}

\email{
  bernparent@gmail.com
}

\department{
  Department of Aerospace Engineering	
}

\institution{
  Pusan National University
}

\title{
  Multidimensional Fluxes
}

\date{
  2019
}

%\setlength\nomenclaturelabelwidth{0.13\hsize}  % optional, default is 0.03\hsize
%\setlength\nomenclaturecolumnsep{0.09\hsize}  % optional, default is 0.06\hsize

\nomenclature{

  \begin{nomenclaturelist}{Roman symbols}
   \item[$a$] speed of sound
  \end{nomenclaturelist}
}


\abstract{
abstract
}

\begin{document}
  \pagestyle{headings}
  \pagenumbering{arabic}
  \setcounter{page}{1}
%%  \maketitle
  \makewarpdoctitle
%  \makeabstract
  \tableofcontents
%  \makenomenclature
%%  \listoftables
%%  \listoffigures


\section{Second-Order Polynomial Fit}

Let's say we wish to find the integral of the flux at the $i+1/2$ interface between two nodes. The integral should be performed from $\eta=-1/2$ to $\eta=1/2$. We know the fluxes $F_{i+1/2,j}$ at $\eta=0$, $F_{i+1/2,j+1}$ at $\eta=1$, and $F_{i+1/2,j-1}$ at $\eta=-1$, as determined through a standard WENO interpolation \cite{jcp:1994:liu}
 of the primitives reconstructed over a FDS \cite{jcp:1981:roe} or FVS flux. Now, let's fit a second-order polynomial through the fluxes:
%
\begin{equation}
F_{i+1/2}=c_1 \eta^2 + c_2 \eta + c_3 
\end{equation}
% 
Find the coefficients $c_1$, $c_2$, $c_3$ by making sure $F_{i+1/2}$ goes through the three $F$s. First, at $\eta=0$, $F_{i+1/2}=F_{i+1/2,j}$:
%
\begin{equation}
c_3=F_{i+1/2,j}
\end{equation}
%
Second, at $\eta=1$, $F_{i+1/2}=F_{i+1/2,j+1}$:
%
\begin{equation}
F_{i+1/2,j+1}=c_1  + c_2  + F_{i+1/2,j} 
\end{equation}
% 
Third, at $\eta=-1$, $F_{i+1/2}=F_{i+1/2,j-1}$:
%
\begin{equation}
F_{i+1/2,j-1}=c_1  - c_2  + F_{i+1/2,j} 
\end{equation}
% 
Add the latter two equations and isolate $c_1$:
%
\begin{equation}
c_1=\frac{F_{i+1/2,j+1}+F_{i+1/2,j-1}}{2}- F_{i+1/2,j} 
\end{equation}
% 
Substitute the latter in the former and isolate $c_2$:
%
\begin{equation}
c_2=\frac{F_{i+1/2,j+1}+F_{i+1/2,j-1}}{2} -F_{i+1/2,j-1}
\end{equation}
% 
Now that $c_1$, $c_2$, and $c_3$ are known, we can integrate the flux from $\eta=-1/2$ to $\eta=1/2$ to obtain the integrated flux $\overline{F}_{i+1/2}$:
%
\begin{equation}
\overline{F}_{i+1/2}=\int_{\eta=-1/2}^{\eta=1/2} F_{i+1/2} d\eta
\end{equation}
%
%
\begin{equation}
\overline{F}_{i+1/2}=\int_{\eta=-1/2}^{\eta=1/2} (c_1 \eta^2 + c_2 \eta + c_3 ) d\eta
\end{equation}
%
Noting that $c_1$, $c_2$, $c_3$ do not depend on $\eta$:
%
\begin{equation}
\overline{F}_{i+1/2}=\left(\frac{1}{3}c_1 \eta^3 + \frac{1}{2} c_2 \eta^2 + c_3 \eta\right)_{\eta=-1/2}^{\eta=1/2} 
\end{equation}
%
or
%
\begin{equation}
\overline{F}_{i+1/2}=\frac{1}{24}c_1  + \frac{1}{8} c_2  + c_3 \frac{1}{2}
+\frac{1}{24}c_1  - \frac{1}{8} c_2  + \frac{1}{2} c_3 
\end{equation}
%
Simplify:
%
\begin{equation}
\overline{F}_{i+1/2}=\frac{1}{12}c_1   + c_3 
\end{equation}
%
Substitute $c_1$ and $c_3$ from previous expressions in the latter:
%
\begin{equation}
\overline{F}_{i+1/2}=\frac{1}{12}\left(\frac{F_{i+1/2,j+1}+F_{i+1/2,j-1}}{2}- F_{i+1/2,j} \right)   + F_{i+1/2,j} 
\end{equation}
%
Simplify:
%
\begin{equation}
\frameeqn{
\overline{F}_{i+1/2}=\frac{1}{24} F_{i+1/2,j+1}+ \frac{1}{24}F_{i+1/2,j-1}   + \frac{11}{12}F_{i+1/2,j} 
}
\end{equation}
%








\appendix


  \bibliographystyle{warpdoc}
  \bibliography{all}


\end{document}






