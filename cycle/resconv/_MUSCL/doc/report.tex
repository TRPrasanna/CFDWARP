\documentclass{warpdoc}
\newlength\lengthfigure                  % declare a figure width unit
\setlength\lengthfigure{0.158\textwidth} % make the figure width unit scale with the textwidth
\usepackage{psfrag}         % use it to substitute a string in a eps figure
\usepackage{subfigure}
\usepackage{rotating}
\usepackage{pstricks}
\usepackage[innercaption]{sidecap} % the cute space-saving side captions
\usepackage{scalefnt}
\usepackage{amsbsy}
\usepackage{bm}
\usepackage{amsmath}

%%%%%%%%%%%%%=--NEW COMMANDS BEGINS--=%%%%%%%%%%%%%%%%%%%%%%%%%%%%%%%%%%
\newcommand{\alb}{\vspace{0.1cm}\\} % array line break
\newcommand{\mfa}{\scriptscriptstyle}
\newcommand{\mfb}{\scriptstyle}
\newcommand{\mfc}{\textstyle}
\newcommand{\mfd}{\displaystyle}
\newcommand{\hlinex}{\vspace{-0.34cm}~~\\ \hline \vspace{-0.31cm}~~\\}
\newcommand{\hlinextop}{\vspace{-0.46cm}~~\\ \hline \hline \vspace{-0.32cm}~~\\}
\newcommand{\hlinexbot}{\vspace{-0.37cm}~~\\ \hline \hline \vspace{-0.50cm}~~\\}
\newcommand{\tablespacing}{\vspace{-0.4cm}}
\newcommand{\fontxfig}{\footnotesize\scalefont{0.918}}
\newcommand{\fontgnu}{\footnotesize\scalefont{0.896}}
\renewcommand{\fontsizetable}{\footnotesize\scalefont{1.0}}
\renewcommand{\fontsizefigure}{\footnotesize}
%\renewcommand{\vec}[1]{\pmb{#1}}
%\renewcommand{\vec}[1]{\boldsymbol{#1}}
\renewcommand{\vec}[1]{\bm{#1}}
\setcounter{tocdepth}{3}
\let\citen\cite
\newcommand\frameeqn[1]{\fbox{$\displaystyle #1$}}

%%%%%%%%%%%%%=--NEW COMMANDS BEGINS--=%%%%%%%%%%%%%%%%%%%%%%%%%%%%%%%%%%

\setcounter{tocdepth}{3}

%%%%%%%%%%%%%=--NEW COMMANDS ENDS--=%%%%%%%%%%%%%%%%%%%%%%%%%%%%%%%%%%%%



\author{
  Bernard Parent
}

\email{
  bernparent@gmail.com
}

\department{
  Department of Aerospace Engineering	
}

\institution{
  Pusan National University
}

\title{
  Flux Integration on Interface
}

\date{
  2019
}

%\setlength\nomenclaturelabelwidth{0.13\hsize}  % optional, default is 0.03\hsize
%\setlength\nomenclaturecolumnsep{0.09\hsize}  % optional, default is 0.06\hsize

\nomenclature{

  \begin{nomenclaturelist}{Roman symbols}
   \item[$a$] speed of sound
  \end{nomenclaturelist}
}


\abstract{
abstract
}

\begin{document}
  \pagestyle{headings}
  \pagenumbering{arabic}
  \setcounter{page}{1}
%%  \maketitle
  \makewarpdoctitle
%  \makeabstract
  \tableofcontents
%  \makenomenclature
%%  \listoftables
%%  \listoffigures


\section{Third Order Central}

Let's say we wish to find the integral of the flux at the $i+1/2$ interface between two nodes. The integral should be performed from $\eta=-1/2$ to $\eta=1/2$. We know the fluxes $F_{i+1/2,j}$ at $\eta=0$, $F_{i+1/2,j+1}$ at $\eta=1$, and $F_{i+1/2,j-1}$ at $\eta=-1$, as determined through a standard WENO interpolation \cite{jcp:1994:liu}
 of the primitives reconstructed over a FDS \cite{jcp:1981:roe} or FVS flux. Now, let's fit a second-order polynomial through the fluxes:
%
\begin{equation}
F_{i+1/2}=c_1 \eta^2 + c_2 \eta + c_3 
\end{equation}
% 
Find the coefficients $c_1$, $c_2$, $c_3$ by making sure $F_{i+1/2}$ goes through the three $F$s. First, at $\eta=0$, $F_{i+1/2}=F_{i+1/2,j}$:
%
\begin{equation}
c_3=F_{i+1/2,j}
\end{equation}
%
Second, at $\eta=1$, $F_{i+1/2}=F_{i+1/2,j+1}$:
%
\begin{equation}
F_{i+1/2,j+1}=c_1  + c_2  + F_{i+1/2,j} 
\end{equation}
% 
Third, at $\eta=-1$, $F_{i+1/2}=F_{i+1/2,j-1}$:
%
\begin{equation}
F_{i+1/2,j-1}=c_1  - c_2  + F_{i+1/2,j} 
\end{equation}
% 
Add the latter two equations and isolate $c_1$:
%
\begin{equation}
c_1=\frac{F_{i+1/2,j+1}+F_{i+1/2,j-1}}{2}- F_{i+1/2,j} 
\end{equation}
% 
Substitute the latter in the former and isolate $c_2$:
%
\begin{equation}
c_2=\frac{F_{i+1/2,j+1}+F_{i+1/2,j-1}}{2} -F_{i+1/2,j-1}
\end{equation}
% 
Now that $c_1$, $c_2$, and $c_3$ are known, we can integrate the flux from $\eta=-1/2$ to $\eta=1/2$ to obtain the integrated flux $\overline{F}_{i+1/2}$:
%
\begin{equation}
\overline{F}_{i+1/2}=\int_{\eta=-1/2}^{\eta=1/2} F_{i+1/2} d\eta
\end{equation}
%
%
\begin{equation}
\overline{F}_{i+1/2}=\int_{\eta=-1/2}^{\eta=1/2} (c_1 \eta^2 + c_2 \eta + c_3 ) d\eta
\end{equation}
%
Noting that $c_1$, $c_2$, $c_3$ do not depend on $\eta$:
%
\begin{equation}
\overline{F}_{i+1/2}=\left(\frac{1}{3}c_1 \eta^3 + \frac{1}{2} c_2 \eta^2 + c_3 \eta\right)_{\eta=-1/2}^{\eta=1/2} 
\end{equation}
%
or
%
\begin{equation}
\overline{F}_{i+1/2}=\frac{1}{24}c_1  + \frac{1}{8} c_2  + c_3 \frac{1}{2}
+\frac{1}{24}c_1  - \frac{1}{8} c_2  + \frac{1}{2} c_3 
\end{equation}
%
Simplify:
%
\begin{equation}
\overline{F}_{i+1/2}=\frac{1}{12}c_1   + c_3 
\end{equation}
%
Substitute $c_1$ and $c_3$ from previous expressions in the latter:
%
\begin{equation}
\overline{F}_{i+1/2}=\frac{1}{12}\left(\frac{F_{i+1/2,j+1}+F_{i+1/2,j-1}}{2}- F_{i+1/2,j} \right)   + F_{i+1/2,j} 
\end{equation}
%
Simplify:
%
\begin{equation}
\frameeqn{
\overline{F}_{i+1/2}=\frac{1}{24} F_{i+1/2,j+1}+ \frac{1}{24}F_{i+1/2,j-1}   + \frac{11}{12}F_{i+1/2,j} 
}
\end{equation}
%



\section{Fifth Order Central}

Let's fit a fourth-order polynomial through the fluxes:
%
\begin{equation}
F_{i+1/2}=c_1 \eta^4 + c_2 \eta^3 + c_3 \eta^2 + c_4 \eta + c_5 
\label{eqn:central5a}
\end{equation}
% 
Find the coefficients $c_1$, $c_2$, $c_3$, $c_4$, and $c_5$ by making sure $F_{i+1/2}$ goes through the three $F$s. First, at $\eta=0$, $F_{i+1/2}=F_{i+1/2,j}$:
%
\begin{equation}
c_5=F_{i+1/2,j}
\label{eqn:central5_c5}
\end{equation}
%
Second, at $\eta=1$, $F_{i+1/2}=F_{i+1/2,j+1}$:
%
\begin{equation}
F_{i+1/2,j+1}=c_1  + c_2  + c_3  + c_4  + c_5 
\label{eqn:central5c}
\end{equation}
% 
Third, at $\eta=-1$, $F_{i+1/2}=F_{i+1/2,j-1}$:
%
\begin{equation}
F_{i+1/2,j-1}=c_1  - c_2  + c_3  - c_4  + c_5 
\label{eqn:central5d}
\end{equation}
% 
Fourth, at $\eta=2$, $F_{i+1/2}=F_{i+1/2,j+2}$:
%
\begin{equation}
F_{i+1/2,j+2}= 16 c_1 + 8 c_2 + 4 c_3 + 2 c_4 + c_5 
\label{eqn:central5e}
\end{equation}
% 
Fifth, at $\eta=-2$, $F_{i+1/2}=F_{i+1/2,j-2}$:
%
\begin{equation}
F_{i+1/2,j-2}= 16 c_1 - 8 c_2 + 4 c_3 - 2 c_4 + c_5 
\label{eqn:central5f}
\end{equation}
% 
Add the former two equations and the latter two equations:
%
\begin{equation}
F_{i+1/2,j-1}+F_{i+1/2,j+1} = 2 c_1    + 2 c_3   + 2 c_5 
\label{eqn:central5g}
\end{equation}
% 
%
\begin{equation}
F_{i+1/2,j-2}+F_{i+1/2,j+2} = 32 c_1  + 8 c_3  + 2 c_5 
\label{eqn:central5h}
\end{equation}
% 
Substitute $c_5=F_{i+1/2,j}$:
%
\begin{equation}
F_{i+1/2,j-1}- 2 F_{i+1/2,j}+F_{i+1/2,j+1} = 2 c_1    + 2 c_3    
\label{eqn:central5i}
\end{equation}
% 
%
\begin{equation}
F_{i+1/2,j-2}- 2 F_{i+1/2,j}+F_{i+1/2,j+2} = 32 c_1  + 8 c_3   
\label{eqn:central5j}
\end{equation}
% 



Multiply Eq.\ (\ref{eqn:central5i}) by -4 and add to Eq.\ (\ref{eqn:central5j}):
%
\begin{equation}
 c_1= \frac{+F_{i+1/2,j-2} -4 F_{i+1/2,j-1}+ 6 F_{i+1/2,j} -4 F_{i+1/2,j+1} +F_{i+1/2,j+2}}{24}   
\label{eqn:central5_c1}
\end{equation}
% 
Multiply Eq.\ (\ref{eqn:central5i}) by -16 and add to Eq.\ (\ref{eqn:central5j}):
%
\begin{equation}
c_3=\frac{+ F_{i+1/2,j-2} -16 F_{i+1/2,j-1}+ 30 F_{i+1/2,j} -16 F_{i+1/2,j+1} 
+F_{i+1/2,j+2}}{24}   
\label{eqn:central5_c3}
\end{equation}
% 

Now that $c_1$, $c_3$, and $c_5$ are known, we can integrate the flux from $\eta=-1/2$ to $\eta=1/2$ to obtain the integrated flux $\overline{F}_{i+1/2}$:
%
\begin{equation}
\overline{F}_{i+1/2}=\int_{\eta=-1/2}^{\eta=1/2} F_{i+1/2} d\eta
\end{equation}
%
or
%
\begin{equation}
\overline{F}_{i+1/2}=\int_{\eta=-1/2}^{\eta=1/2} \left( c_1 \eta^4 + c_2 \eta^3 + c_3 \eta^2 + c_4 \eta + c_5 \right) d\eta
\end{equation}
%
Noting that $c_1$, $c_2$, $c_3$ do not depend on $\eta$:
%
\begin{equation}
\overline{F}_{i+1/2}= \left[ \frac{c_1}{5} \eta^5 + \frac{c_2}{4} \eta^4 + \frac{c_3}{3} \eta^3 + \frac{c_4}{2} \eta^2 + c_5 \eta \right]_{\eta=-1/2}^{\eta=1/2}
\end{equation}
%
or
%
\begin{equation}
\overline{F}_{i+1/2}=  \frac{c_1}{80}  + \frac{c_3}{12}   + c_5  
\end{equation}
%
Substitute $c_1$ and $c_3$ in the latter:
%
\begin{align}
\overline{F}_{i+1/2}= & \frac{+F_{i+1/2,j-2} -4 F_{i+1/2,j-1}+ 6 F_{i+1/2,j} -4 F_{i+1/2,j+1} +F_{i+1/2,j+2}}{24\times 80}\nonumber\\
 &+ \frac{+ F_{i+1/2,j-2} -16 F_{i+1/2,j-1}+ 30 F_{i+1/2,j} -16 F_{i+1/2,j+1} +F_{i+1/2,j+2}}{24\times 12}  
 + F_{i+1/2,j} 
\end{align}
%
Put all terms under a common denominator:
%
\begin{align}
\overline{F}_{i+1/2}= & \frac{+3 F_{i+1/2,j-2} -12 F_{i+1/2,j-1}+ 18 F_{i+1/2,j} - 12 F_{i+1/2,j+1} +3F_{i+1/2,j+2}}{24\times 240}\nonumber\\
 &+ \frac{+ 20 F_{i+1/2,j-2} -320 F_{i+1/2,j-1}+ 600 F_{i+1/2,j} -320 F_{i+1/2,j+1} +20 F_{i+1/2,j+2}}{24\times 240}  
 + \frac{ 5760 F_{i+1/2,j}}{24\times 240} 
\end{align}
%
Regroup:
%
\begin{equation}
\frameeqn{
\overline{F}_{i+1/2}=  \frac{+23 F_{i+1/2,j-2} -332 F_{i+1/2,j-1}+ 6378 F_{i+1/2,j} - 332 F_{i+1/2,j+1} + 23 F_{i+1/2,j+2}}{5760}
}
\end{equation}
%
Also, can find $c_2$ and $c_4$ by subtracting (\ref{eqn:central5d}) from (\ref{eqn:central5c}) and subtracting (\ref{eqn:central5f}) from (\ref{eqn:central5e}):
%
\begin{equation}
F_{i+1/2,j+1}-F_{i+1/2,j-1}=2 c_2    + 2 c_4   
\label{eqn:central5k}
\end{equation}
% 
%
\begin{equation}
F_{i+1/2,j+2}-F_{i+1/2,j-2}=  16 c_2  + 4 c_4  
\label{eqn:central5l}
\end{equation}
% 
Multiply (\ref{eqn:central5k}) by $-2$ and add it to (\ref{eqn:central5l}):
%
\begin{equation}
c_2=\frac{F_{i+1/2,j+2}-F_{i+1/2,j-2}-2F_{i+1/2,j+1}+2F_{i+1/2,j-1}}{12}    
\label{eqn:central5_c2}
\end{equation}
% 
Multiply (\ref{eqn:central5k}) by $-8$ and add it to (\ref{eqn:central5l}):
%
\begin{equation}
c_4=\frac{-8F_{i+1/2,j+1}+8F_{i+1/2,j-1}+F_{i+1/2,j+2}-F_{i+1/2,j-2}}{-12}  
\end{equation}
% 



\section{3rd-5th Adaptive Order WENO}

First find the indicator of smoothness for the 3rd order and 5th order central stencils. We define the indicator of smoothness $\beta$  as follows:
%
\begin{equation}
\beta = \sum_m (\Delta \eta)^{2m-1} \int_{\eta=-0.5}^{\eta=0.5} \left( \frac{d^m F_{i+1/2}}{d\eta^m} \right)^2
\end{equation}
%
But $\Delta \eta=1$:
%
\begin{equation}
\beta = \sum_m  \int_{\eta=-0.5}^{\eta=0.5} \left( \frac{d^m F_{i+1/2}}{d\eta^m} \right)^2
\end{equation}
%




\appendix


  \bibliographystyle{warpdoc}
  \bibliography{all}


\end{document}






