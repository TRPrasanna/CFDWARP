\documentclass{warpdoc}
\newlength\lengthfigure                  % declare a figure width unit
\setlength\lengthfigure{0.158\textwidth} % make the figure width unit scale with the textwidth
\usepackage{psfrag}         % use it to substitute a string in a eps figure
\usepackage{subfigure}
\usepackage{rotating}
\usepackage{pstricks}
\usepackage[innercaption]{sidecap} % the cute space-saving side captions
\usepackage{scalefnt}
\usepackage{bm}
\usepackage{amsmath}

%%%%%%%%%%%%%=--NEW COMMANDS BEGINS--=%%%%%%%%%%%%%%%%%%%%%%%%%%%%%%%%%%
\newcommand{\alb}{\vspace{0.2cm}\\} % array line break
\newcommand{\efficiency}{\eta}
\newcommand{\ordi}{{\rm d}}
\newcommand{\unitvecdiff}[2]{\overline{\vec{#1} - \vec{#2}}}
%\let\vec\bf
\renewcommand{\vec}[1]{\bm{#1}}
\newcommand{\rhos}{\rho}
\newcommand{\Cv}{{C_{\rm v}}}
\newcommand{\Cp}{{C_{\rm p}}}
\newcommand{\Sct}{{{\rm Sc}_{\rm T}}}
\newcommand{\Prt}{{{\rm Pr}_{\rm T}}}
\newcommand{\nd}{{{n}_{\rm d}}}
\newcommand{\ns}{{{n}_{\rm s}}}
\newcommand{\nn}{{{n}_{\rm n}}}
\newcommand{\nr}{{{n}_{\rm r}}}
\newcommand{\ndm}{{\bar{n}_{\rm d}}}
\newcommand{\nsm}{{\bar{n}_{\rm s}}}
\newcommand{\turb}{_{\rm T}}
\newcommand{\mut}{{\mu\turb}}
\newcommand{\mfa}{\scriptscriptstyle}
\newcommand{\mfb}{\scriptstyle}
\newcommand{\mfc}{\textstyle}
\newcommand{\mfd}{\displaystyle}
\newcommand{\hlinex}{\vspace{-0.34cm}~~\\ \hline \vspace{-0.31cm}~~\\}
\newcommand{\hlinextop}{\vspace{-0.46cm}~~\\ \hline \hline \vspace{-0.32cm}~~\\}
\newcommand{\hlinexbot}{\vspace{-0.37cm}~~\\ \hline \hline \vspace{-0.50cm}~~\\}
\newcommand{\tablespacing}{\vspace{-0.4cm}}
\newcommand{\fontxfig}{\footnotesize\scalefont{0.918}}
\newcommand{\fontgnu}{\footnotesize\scalefont{0.896}}
\renewcommand{\fontsizetable}{\footnotesize\scalefont{0.9}}
\setcounter{tocdepth}{3}
\let\citen\cite

%%%%%%%%%%%%%=--NEW COMMANDS ENDS--=%%%%%%%%%%%%%%%%%%%%%%%%%%%%%%%%%%%%
%%%%%%%%%%%%%=--NEW COMMANDS BEGINS--=%%%%%%%%%%%%%%%%%%%%%%%%%%%%%%%%%%


\author{
  Bernard Parent 
}

\email{
  bernparent@gmail.com
}

\department{
  Aerospace and Mechanical Engineering
}

\institution{
  University of Arizona
}

\title{Air Plasma Kinetics at High Temperature and Electric Field
}

\date{
  December 2019
}

%\setlength\nomenclaturelabelwidth{0.13\hsize}  % optional, default is 0.03\hsize
%\setlength\nomenclaturecolumnsep{0.09\hsize}  % optional, default is 0.06\hsize

\nomenclature{

  \begin{nomenclaturelist}{Roman symbols}
   \item[$a$] speed of sound
  \end{nomenclaturelist}
}


\abstract{
abstract
}

\begin{document}
  \pagestyle{headings}
  \pagenumbering{arabic}
  \setcounter{page}{1}
%%  \maketitle
  \makewarpdoctitle
%  \makeabstract
  \tableofcontents
%  \makenomenclature
  \listoftables
%%  \listoffigures



\section{Chemical Models in Arrhenius Form}






%
\begin{table}[t]
\fontsizetable
\begin{center}
\begin{threeparttable}
\tablecaption{Dunn-Kang 11-species 31-reaction high-temperature air model \cite{nasa:1973:dunn,aiaaconf:1987:Bussing}.}
\begin{tabular}{ccccc} 
\toprule
\multicolumn{2}{c}{Reaction} & $A$, $\textrm{cm}^3\cdot(\textrm{mole}\cdot \textrm{s})^{-1}\cdot \textrm{K}^{-n}$ & $n$ & $E$, cal/mole  \\ 
\midrule
(1) & $\rm O_2 + N \rightleftarrows 2O+N$ & 3.6 $\times$ 10$^{18}$  & -1 & 118,800 \\
(2) & $\rm O_2 + NO \rightleftarrows 2O+NO$ & 3.6 $\times$ 10$^{18}$ & -1 & 118,800 \\
(3) & $\rm N_2 + O \rightleftarrows 2N+O$ & 1.9 $\times$ 10$^{17}$ & -0.5 & 226,000 \\
(4) & $\rm N_2 + NO \rightleftarrows 2N+NO$ & 1.9 $\times$ 10$^{17}$ & -0.5 & 226,000 \\
(5) & $\rm N_2 + O_2 \rightleftarrows 2N+O_2$ & 1.9 $\times$ 10$^{17}$ & -0.5 & 226,000 \\
(6) & $\rm NO + O_2 \rightleftarrows N+O+O_2$ & 3.9 $\times$ 10$^{20}$ & -1.5 & 151,000 \\
(7) & $\rm NO + N_2 \rightleftarrows N+O+N_2$ & 3.9 $\times$ 10$^{20}$ & -1.5 & 151,000 \\
(8) & $\rm O + NO \rightleftarrows N+O_2$ & 3.2 $\times$ 10$^{9}$ & 1 & 39,400 \\
(9) & $\rm O + N_2 \rightleftarrows N+NO$ & 7 $\times$ 10$^{13}$ & 0 & 76,000 \\
(10) & $\rm N + N_2 \rightleftarrows 2N+N$ & 4.085 $\times$ 10$^{22}$ & -1.5 & 226,000 \\
(11) & $\rm O + N \rightleftarrows NO^{+}+e^{-}$ & 1.4 $\times$ 10$^{6}$ & 1.5 & 63,800 \\
(12) & $\rm O + e^- \rightleftarrows O^++2e^-$ & 3.6 $\times$ 10$^{31}$ & -2.91 & 316,000 \\
(13) & $\rm N + e^- \rightleftarrows N^++2e^-$ & 1.1 $\times$ 10$^{32}$ & -3.14 & 338,000 \\
(14) & $\rm O + O \rightleftarrows O_2^{+}+e^-$ & 1.6 $\times$ 10$^{17}$ & -0.98 & 161,600 \\
(15) & $\rm O + O_2^+ \rightleftarrows O_2+O^+$ & 2.92 $\times$ 10$^{18}$ & -1.11 & 56,000 \\
(16) & $\rm N_2 + N^+ \rightleftarrows N+N_2^{+}$ & 2.02 $\times$ 10$^{11}$ & 0.81 & 26,000 \\
(17) & $\rm N + N \rightleftarrows N_2^{+} + e^-$ & 1.4 $\times$ 10$^{13}$ & 0 & 135,600 \\
(18) & $\rm O + NO^+ \rightleftarrows NO + O^+$ & 3.63 $\times$ 10$^{15}$ & -0.6 & 101,600 \\
(19) & $\rm N_2 + O^+ \rightleftarrows O + N_2^{+}$ & 3.4 $\times$ 10$^{19}$ & -2 & 46,000 \\
(20) & $\rm N + NO^+ \rightleftarrows NO + N^+$ & 1 $\times$ 10$^{19}$ & -0.93 & 122,000 \\
(21) & $\rm O_2 + NO^+ \rightleftarrows NO + O_2^{+}$ & 1.8 $\times$ 10$^{15}$ & 0.17 & 66,000 \\
(22) & $\rm O + NO^+ \rightleftarrows O_2 + N^+$ & 1.34 $\times$ 10$^{13}$ & 0.31 & 154,540 \\
(23) & $\rm O_2 + O \rightleftarrows 2O + O$ & 9 $\times$ 10$^{19}$ & -1 & 119,000 \\
(24) & $\rm O_2 + O_2 \rightleftarrows 2O + O_2$ & 3.24 $\times$ 10$^{19}$ & -1 & 119,000 \\
(25) & $\rm O_2 + N_2 \rightleftarrows 2O + N_2$ & 7.2 $\times$ 10$^{18}$ & -1 & 119,000 \\
(26) & $\rm N_2 + N_2 \rightleftarrows 2N + N_2$ & 4.7 $\times$ 10$^{17}$ & -0.5 & 226,000 \\
(27) & $\rm NO + O \rightleftarrows N + 2O$ & 7.8 $\times$ 10$^{20}$ & -1.5 & 151,000 \\
(28) & $\rm NO + N \rightleftarrows O + 2N$ & 7.8 $\times$ 10$^{20}$ & -1.5 & 151,000 \\
(29) & $\rm NO + NO \rightleftarrows N + O + NO$ & 7.8 $\times$ 10$^{20}$ & -1.5 & 151,000 \\
(30) & $\rm O2 + N2 \rightleftarrows NO + NO^+ + e^-$ & 1.38 $\times$ 10$^{20}$ & -1.84 & 282,000 \\
(31) & $\rm NO + N2 \rightleftarrows NO^+ + e^- + N_2$ & 2.2 $\times$ 10$^{15}$ & -0.35 & 216,000 \\
\bottomrule
\end{tabular}
\label{tab:dunn-kang}
\end{threeparttable}
\end{center}
\end{table}
%











%
\begin{table}
  \center\fontsizetable
  \begin{threeparttable}
    \tablecaption{Macheret Townsend Ionization and Electron Attachment Reactions in Air.\tnote{a}}
    \label{tab:macheret}
    \fontsizetable
    \begin{tabular*}{\textwidth}{l@{\extracolsep{\fill}}lll}
    \toprule
    No.&Reaction & Rate Coefficient  & Refs. \\
    \midrule
    32  & $\rm e^- + N_2   \rightarrow N_2^+ + e^- + e^-$  
       &  ${\rm exp}(-0.0105809\cdot {\rm ln}^2 E^\star - 2.40411\cdot 10^{-75} \cdot {\rm ln}^{46}E^\star)$~cm$^3$/s
       & \cite{jcp:2014:parent} \\
    33  & $\rm e^- + O_2   \rightarrow O_2^+ + e^- + e^-$  
       &  ${\rm exp}(-0.0102785\cdot {\rm ln}^2 E^\star - 2.42260\cdot 10^{-75} \cdot {\rm ln}^{46}E^\star)$~cm$^3$/s
       & \cite{jcp:2014:parent} \\
    34 & $\rm O_2^{-}+N_2^{+} \rightarrow O_2 + N_2$ 
       & $2.0 \cdot 10^{-7} \cdot (300/T)^{0.5}$ cm$^3$/s
       & \cite{misc:1992:kossyi}\\
    35 & $\rm O_2^{-}+O_2^{+} \rightarrow O_2 + O_2$ 
       & $2.0 \cdot 10^{-7} \cdot (300/T)^{0.5}$ cm$^3$/s
       & \cite{misc:1992:kossyi}\\
    36 & $\rm O_2^{-}+N_2^{+} + N_2\rightarrow O_2 + N_2 +N_2$ 
       & $2.0 \cdot 10^{-25} \cdot (300/T)^{2.5}$ cm$^6$/s  
       & \cite{misc:1992:kossyi}\\
    37 & $\rm O_2^{-}+O_2^{+} + N_2\rightarrow O_2 + O_2 +N_2$ 
       & $2.0 \cdot 10^{-25} \cdot (300/T)^{2.5}$ cm$^6$/s  
       & \cite{misc:1992:kossyi}\\
    38 & $\rm O_2^{-}+N_2^{+} + O_2\rightarrow O_2 + N_2 +O_2$ 
       & $2.0 \cdot 10^{-25} \cdot (300/T)^{2.5}$ cm$^6$/s  
       & \cite{misc:1992:kossyi}\\
    39 & $\rm O_2^{-}+O_2^{+} + O_2\rightarrow O_2 + O_2 +O_2$ 
       & $2.0 \cdot 10^{-25} \cdot (300/T)^{2.5}$ cm$^6$/s  
       & \cite{misc:1992:kossyi}\\
    40 & $\rm e^- + O_2 +O_2 \rightarrow O_2^- + O_2$  
       &  $1.4 \cdot 10^{-29} \cdot \left( {300}/{T_{\rm e}}\right)\cdot  \exp \left( {-600}/{T}\right)$
       & \cite{misc:1992:kossyi}\\
    ~  &   
       & ~~~$\cdot \exp \left( {700 \cdot (T_{\rm e}-T)}/{(T_{\rm e} T)}  \right)$ cm$^6$/s
       & ~\\
    41 & $\rm e^- + O_2 + N_2 \rightarrow O_2^- + N_2$  
       & $1.07 \cdot 10^{-31} \cdot \left( {300}/{T_{\rm e}} \right)^2 \cdot \exp \left( {-70}/{T}\right)$          
       & \cite{misc:1992:kossyi}\\
    ~  &   
       & ~~~$\cdot \exp \left( {1500 \cdot (T_{\rm e}-T)}/({T_{\rm e} T})  \right)$ cm$^6$/s 
       & ~\\
    42  & $\rm O_2^- + O_2 \rightarrow e^- + O_2 + O_2$  
       & $8.6 \cdot 10^{-10} \cdot \exp \left( {-6030}/{T}\right)
               \left(1-\exp \left( {-1570}/{T} \right)  \right)$ cm$^3$/s
       & \cite{book:1997:bazelyan}, Ch.\ 2\\
    \bottomrule
    \end{tabular*}
\begin{tablenotes}
\item[{a}] Notation and units: $E^\star$ is the reduced effective electric field in the electron reference frame ($E^\star\equiv|\vec{E}+\vec{V}^{\rm e}\times \vec{B}|/N$) in units of V$\cdot$m$^2$; $T_{\rm e}$ is the electron temperature in Kelvin; $T$ is the neutrals temperature in Kelvin.
\end{tablenotes}
   \end{threeparttable}
\end{table}
%



\bibliographystyle{warpdoc}
\bibliography{all}


\end{document}



