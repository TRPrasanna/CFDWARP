\documentclass{warpdoc}
\newlength\lengthfigure                  % declare a figure width unit
\setlength\lengthfigure{0.158\textwidth} % make the figure width unit scale with the textwidth
\usepackage{psfrag}         % use it to substitute a string in a eps figure
\usepackage{subfigure}
\usepackage{rotating}
\usepackage{pstricks}
\usepackage[innercaption]{sidecap} % the cute space-saving side captions
\usepackage{scalefnt}
\usepackage{bm}
\usepackage{amsmath}

%%%%%%%%%%%%%=--NEW COMMANDS BEGINS--=%%%%%%%%%%%%%%%%%%%%%%%%%%%%%%%%%%
\newcommand{\alb}{\vspace{0.2cm}\\} % array line break
\newcommand{\efficiency}{\eta}
\newcommand{\ordi}{{\rm d}}
\newcommand{\unitvecdiff}[2]{\overline{\vec{#1} - \vec{#2}}}
%\let\vec\bf
\renewcommand{\vec}[1]{\bm{#1}}
\newcommand{\rhos}{\rho}
\newcommand{\Cv}{{C_{\rm v}}}
\newcommand{\Cp}{{C_{\rm p}}}
\newcommand{\Sct}{{{\rm Sc}_{\rm T}}}
\newcommand{\Prt}{{{\rm Pr}_{\rm T}}}
\newcommand{\nd}{{{n}_{\rm d}}}
\newcommand{\ns}{{{n}_{\rm s}}}
\newcommand{\nn}{{{n}_{\rm n}}}
\newcommand{\nr}{{{n}_{\rm r}}}
\newcommand{\ndm}{{\bar{n}_{\rm d}}}
\newcommand{\nsm}{{\bar{n}_{\rm s}}}
\newcommand{\turb}{_{\rm T}}
\newcommand{\mut}{{\mu\turb}}
\newcommand{\mfa}{\scriptscriptstyle}
\newcommand{\mfb}{\scriptstyle}
\newcommand{\mfc}{\textstyle}
\newcommand{\mfd}{\displaystyle}
\newcommand{\hlinex}{\vspace{-0.34cm}~~\\ \hline \vspace{-0.31cm}~~\\}
\newcommand{\hlinextop}{\vspace{-0.46cm}~~\\ \hline \hline \vspace{-0.32cm}~~\\}
\newcommand{\hlinexbot}{\vspace{-0.37cm}~~\\ \hline \hline \vspace{-0.50cm}~~\\}
\newcommand{\tablespacing}{\vspace{-0.4cm}}
\newcommand{\fontxfig}{\footnotesize\scalefont{0.918}}
\newcommand{\fontgnu}{\footnotesize\scalefont{0.896}}
\renewcommand{\fontsizetable}{\footnotesize\scalefont{0.9}}
\setcounter{tocdepth}{3}
\let\citen\cite

%%%%%%%%%%%%%=--NEW COMMANDS ENDS--=%%%%%%%%%%%%%%%%%%%%%%%%%%%%%%%%%%%%
%%%%%%%%%%%%%=--NEW COMMANDS BEGINS--=%%%%%%%%%%%%%%%%%%%%%%%%%%%%%%%%%%


\author{
  Ajjay Omprakas 
}

\email{
  ajjayomprakas@gmail.com
}

\department{
  Aerospace and Mechanical Engineering
}

\institution{
  University of Arizona
}

\title{Hydrogen - Air High pressure combustion reaction mechanism
}

\date{
  June 2020
}

%\setlength\nomenclaturelabelwidth{0.13\hsize}  % optional, default is 0.03\hsize
%\setlength\nomenclaturecolumnsep{0.09\hsize}  % optional, default is 0.06\hsize

\nomenclature{

  \begin{nomenclaturelist}{Roman symbols}
   \item[$a$] speed of sound
  \end{nomenclaturelist}
}


\abstract{
abstract
}

\begin{document}
  \pagestyle{headings}
  \pagenumbering{arabic}
  \setcounter{page}{1}
%%  \maketitle
  \makewarpdoctitle
%  \makeabstract
%  \tableofcontents
%  \makenomenclature
%  \listoftables
%%  \listoffigures



%\section{Chemical Models in Arrhenius Form}






%
\begin{table}[t]
\fontsizetable
\begin{center}
\begin{threeparttable}
\tablecaption{Hydrogen - Air High pressure combustion reaction mechanism \cite{GRImech}.}
\begin{tabular}{ccccc} 
\toprule
\multicolumn{2}{c}{Reaction} & $A$, $\textrm{cm}^3\cdot(\textrm{mole}\cdot \textrm{s})^{-1}\cdot \textrm{K}^{-n}$ & $n$ & $E$, cal/mole  \\ 
\midrule
    1$^a$ & $\rm 2O + M2 \rightleftarrows  O_2 + M2$&$1.2 \times 10^{17}$& -1.0 &  0\\
    2$^a$ & $\rm O +  H + M2 \rightleftarrows  OH + M2$  & $ 5 \times 10^{17} $& -1.0 &  0\\
    3& $\rm O +  H_2  \rightleftarrows  OH + H$  & $ 3.87 \times 10^{4} $& 2.7 & 6260 \\
    4 & $\rm O +  HO_2  \rightleftarrows  OH + O_2$  & $ 2.0 \times 10^{13} $& 0 & 0 \\
    5 & $\rm O +  H_2O_2  \rightleftarrows  OH + HO_2$  & $ 9.63 \times 10^{6} $& 2.0 & 4000  \\
    6$^b$ & $\rm H + O_2 + M1  \rightleftarrows  HO_2 + M1$  & $ 2.8 \times 10^{18} $& -0.862 & 0 \\
    7 & $\rm H + 2O_2  \rightleftarrows  HO_2 + O_2$  & $ 2.8 \times 10^{19} $& -1.24 & 0 \\
    8 & $\rm H + O_2 + H_2O  \rightleftarrows  HO_2 + H_2O$  & $ 11.26 \times 10^{18} $& -0.76 & 0 \\
    9 & $\rm H + O_2 + N_2  \rightleftarrows  HO_2 + N_2$  & $ 2.6 \times 10^{19} $& -1.24 & 0 \\
    10 & $\rm H + O_2 \rightleftarrows  OH + O$  & $ 2.65 \times 10^{16} $& -0.6707 & 17041 \\
    11$^a$ & $\rm 2H + M2 \rightleftarrows  H_2 + M2$  & $ 1.0 \times 10^{18} $& -1.0 & 0 \\
    12 & $\rm 2H + H_2 \rightleftarrows  2H_2 $  & $ 9.0 \times 10^{16} $& -0.6 & 0 \\
    13 & $\rm 2H + H_2O \rightleftarrows  H_2 + H_2O$  & $ 6.0 \times 10^{19} $& -1.25 & 0 \\
    14$^a$ & $\rm H + OH + M2 \rightleftarrows  H_2O + M2$  & $ 2.2 \times 10^{22} $& -2.0 & 0 \\
    15 & $\rm H + HO_2 \rightleftarrows  H_2O + O$  & $ 3.97 \times 10^{12} $& 0 & 671 \\
    16 & $\rm H + HO_2 \rightleftarrows  OH + OH$  & $ 0.84 \times 10^{14} $& 0 & 635 \\
    17 & $\rm H + H_2O_2 \rightleftarrows  HO_2 + H_2$  & $ 1.2 \times 10^{7} $& 2.0 & 5200 \\
    18 & $\rm H + H_2O_2 \rightleftarrows  OH + H_2O$  & $ 1 \times 10^{13} $& 0 & 3600 \\
    19 & $\rm OH + H_2 \rightleftarrows  H + H_2O$  & $ 2.16 \times 10^{08} $& 1.510 & 3430 \\
    20$^a$ & $\rm 2OH + M2 \rightleftarrows  H_2O_2 + M2$  & $ 7.4 \times 10^{13} $& -0.371 & 0 \\
    21 & $\rm 2OH \rightleftarrows  H_2O + O$  & $ 3.57 \times 10^{04} $& 2.4 & -2100 \\
    22 & $\rm OH + HO_2 \rightleftarrows  H_2O + O_2$  & $ 1.45 \times 10^{13} $& 0 & -500 \\
    23 & $\rm OH + H_2O_2 \rightleftarrows  H_2O + HO_2$  & $ 2.0 \times 10^{12} $& 0 & 427 \\
    24 & $\rm OH + H_2O_2 \rightleftarrows  H_2O + HO_2$  & $ 1.7 \times 10^{18} $& 0 & 29410  \\
    25 & $\rm HO_2 + HO_2 \rightleftarrows  O_2 + H_2O_2$  & $ 1.3 \times 10^{11} $& 0 & -1630  \\
    26 & $\rm HO_2 + HO_2 \rightleftarrows  O_2 + H_2O_2$  & $ 4.2 \times 10^{14} $& 0 & 12000  \\
    27 & $\rm OH + HO_2 \rightleftarrows  O_2 + H_2O$  & $ 0.5 \times 10^{16} $& -1.0 & 17330  \\
\bottomrule
\end{tabular}
\label{tab:smith-reaction}
\begin{tablenotes}
\item[{a}] The reaction includes third body were M1 is taken to be H$_2$O, N$_2$, O$_2$
\item[{b}] The reaction includes third body were M2 is taken to be H$_2$O, N$_2$, H$_2$\\
\end{tablenotes}
\end{threeparttable}
\end{center}
\end{table}
%












\bibliographystyle{warpdoc}
\bibliography{all}


\end{document}



